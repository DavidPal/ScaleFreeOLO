\section{Introduction}
\label{section:introduction}

Online Linear Optimization (OLO) is a problem where an algorithm repeatedly
chooses a point $w_t$ from a convex decision set $K$, observes an arbitrary, or
even adversarially chosen, loss vector $\ell_t$ and suffers loss $\langle
\ell_t, w_t \rangle$.  The goal of the algorithm is to have a small cumulative
loss. Performance of an algorithm is evaluated by the so-called regret, which
is the difference of cumulative losses of the algorithm and of the
(hypothetical) strategy that would choose in every round the same best point in
hindsight.

OLO is a fundamental problem in machine
learning~\cite{Cesa-Bianchi-Lugosi-2006,Shalev-Shwartz-2011}.  Many learning
problems can be directly phrased as OLO, e.g., learning with expert
advice~\cite{Littlestone-Warmuth-1994,Vovk-1998,Cesa-Bianchi-Haussler-Helmbold-Schapire-Warmuth-1997}, online combinatorial
optimization~\cite{Koolen-Warmuth-Kivinen-2010}. Other
problems can be reduced to OLO, e.g. online convex
optimization~\cite[Chapter~2]{Shalev-Shwartz-2011}, online classification and
regression~\cite[Chapters~11~and~12]{Cesa-Bianchi-Lugosi-2006}, multi-armed
problems~\cite[Chapter~6]{Cesa-Bianchi-Lugosi-2006}, and batch and stochastic
optimization of convex functions~\cite{Nemirovski-Yudin-1983}.  Hence, a result in OLO
immediately implies other results in all these domains.

The adversarial choice of the loss vectors received by the algorithm is what
makes the OLO problem challenging. In particular, if an OLO algorithm commits
to an upper bound on the norm of future loss vectors, its regret can be made
arbitrarily large through an adversarial strategy that produces loss vectors
with norms that exceed the upper bound.

For this reason, most of the existing OLO algorithms receive as an input---or
explicitly assume---an upper bound $B$ on the norm of the loss vectors.  The
input $B$ is often disguised as the learning rate, the regularization
parameter, or the parameter of strong convexity of the regularizer. Examples of
such algorithms include the Hedge algorithm or online projected gradient
descent with fixed learning rate.  However, these algorithms have two obvious
drawbacks.

First, they do not come with any regret guarantee for sequences of loss vectors
with norms exceeding $B$. Second, on sequences where the norm of loss vectors
is bounded by $b \ll B$, these algorithms fail to have an optimal regret
guarantee that depends on $b$ rather than on $B$.

% Change spacing between rows.
\renewcommand{\arraystretch}{1.5}

\begin{table}[t]
\fontsize{8}{8.2}\selectfont
\centering
\begin{tabular}{|p{3.6cm}|c|p{3.4cm}|c|}
\hline
\textbf{Algorithm} & \textbf{Decisions Set(s)} & \textbf{Regularizer(s)} & \textbf{Scale-Free} \\ \hline \hline
\textsc{Hedge} \cite{Freund-Schapire-1999} & Probability Simplex & Negative Entropy & No \\ \hline
\textsc{GIGA} \cite{Zinkevich-2003} & Any Bounded & $\frac{1}{2}\|w\|_2^2$ & No \\ \hline
\textsc{RDA} \cite{Xiao-2010} & \textbf{Any} & \textbf{Any Strongly Convex} & No \\ \hline
\textsc{FTRL-Proximal} \cite{McMahan-Streeter-2010,McMahan-2014} & Any Bounded & $\frac{1}{2}\|w\|_2^2 + $ any convex func. & \textbf{Yes} \\ \hline
\textsc{AdaGrad MD} \cite{Duchi-Hazan-Singer-2011} & Any Bounded & $\frac{1}{2}\|w\|_2^2 + $ any convex func. & \textbf{Yes} \\ \hline
\textsc{AdaGrad FTRL} \cite{Duchi-Hazan-Singer-2011} & \textbf{Any} & $\frac{1}{2}\|w\|_2^2 + $ any convex func. & No \\ \hline
\textsc{AdaHedge} \cite{de-Rooij-van-Erven-Grunwald-Koolen-2014} & Probability Simplex & Negative Entropy & \textbf{Yes} \\ \hline
\textsc{Optimistic MD} \cite{Rakhlin-Sridharan-2013} & $\sup_{u,v \in K} \Breg_f(u,v) < \infty$ & \textbf{Any Strongly Convex} & \textbf{Yes} \\ \hline
\textsc{NAG} \cite{Ross-Mineiro-Langford-2013} & $\{u: \max_t \langle \ell_t, u\rangle \leq C\}$ & $\frac{1}{2}\|w\|_2^2 $& Partially\footnotemark \\ \hline
\textsc{Scale invariant algorithms} \cite{Orabona-Crammer-Cesa-Bianchi-2014} & \textbf{Any} & $\frac{1}{2}\|w\|_p^2 + $ any convex func. \newline $1<p\leq2$ & Partially\textsuperscript{\ref{footnote-label}} \\ \hline
\textsc{AdaFTRL} \textbf{[this paper]} & Any Bounded & \textbf{Any Strongly Convex} & \textbf{Yes} \\ \hline
\textsc{SOLO FTRL} \textbf{[this paper]} & \textbf{Any} & \textbf{Any Strongly Convex} & \textbf{Yes} \\ \hline
\end{tabular}
\caption{Selected results for OLO. Best results in each column are in bold.
\label{table:results}
}
\end{table}

\footnotetext{\label{footnote-label} These algorithms attempt to produce an invariant sequence of predictions $\langle w_t, \ell_t \rangle$, rather than a sequence of invariant $w_t$.}


There is a clear practical need to design algorithms that adapt automatically
to norms of the loss vectors.  A natural, yet overlooked, design method to
achieve this type of adaptivity is by insisting to have a \textbf{scale-free}
algorithm.  That is, the sequence of decisions of the algorithm does not change
if the sequence of loss vectors is multiplied by a positive constant.

A summary of algorithms for OLO is presented in Table~\ref{table:results}.
While the scale-free property has been looked at in the expert setting, in the
general OLO setting this issue has been largely ignored.  In particular, the
AdaHedge~\cite{de-Rooij-van-Erven-Grunwald-Koolen-2014} algorithm, for
prediction with expert advice, is specifically designed to be scale-free.  A
notable exception in the OLO literature is the discussion of the ``off-by-one''
issue in~\cite{McMahan-2014}, where it is explained that even the popular AdaGrad
algorithm~\cite{Duchi-Hazan-Singer-2011} is not completely adaptive; see also
our discussion in Section~\ref{section:solo-ftrl}. In particular, existing
scale-free algorithms cover only some norms/regularizers and \emph{only}
bounded decision sets. The case of \textbf{unbounded decision sets},
practically the most interesting one for machine learning applications, remains
completely unsolved.

Rather than trying to design strategies for a particular form of loss
vectors and/or decision sets, in this paper we explicitly focus on the
scale-free property. Regret of scale-free algorithms is proportional to the
scale of the losses, ensuring optimal linear dependency on the maximum norm of
the loss vectors.

The contribution of this paper is twofold. First, in
Section~\ref{section:ada-ftrl} we show that the analysis and design of
AdaHedge can be generalized to the OLO scenario and to any strongly convex
regularizer, in an algorithm we call \textsc{AdaFTRL}, providing a new and
rather interesting way to adapt the learning rates to have scale-free
algorithms.  Second, in Section~\ref{section:solo-ftrl} we propose a new and
simple algorithm, \textsc{SOLO FTRL}, that is scale-free and is the
\textbf{first} scale-free online algorithm for unbounded sets with a
non-vacuous regret bound.  Both algorithms are instances of Follow The
Regularized Leader (FTRL) with an adaptive learning rate.  Moreover, our
algorithms show that scale-free algorithms can be obtained in a ``native'' and
simple way, i.e.  without using ``doubling tricks'' that attempt to fix poorly
designed algorithms rather than directly solving the problem.

For both algorithms, we prove that for bounded decision sets the regret after
$T$ rounds is at most $O(\sqrt{\sum_{t=1}^T\|\ell_t\|_*^2})$.  We show that the
$\sqrt{\sum_{t=1}^T \|\ell_t\|_*^2}$ term is necessary by proving a $\Omega (D
\sqrt{\sum_{t=1}^T\|\ell_t\|_*^2} )$ lower bound on the regret of any algorithm
for OLO for any decision set with diameter $D$ with respect to the primal norm
$\|\cdot\|$. For the \textsc{SOLO FTRL} algorithm, we prove an
$O(\max_{t=1,2,\dots,T} \|\ell_t\|_* \sqrt{T})$ regret bound for any unbounded
decision set.

Our algorithms are also \textbf{any-time}, i.e., do not need to know the
number of rounds in advance and our regret bounds hold for all time steps
simultaneously.
