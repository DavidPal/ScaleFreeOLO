\section{Introduction}
\label{section:introduction}

Online Linear Optimization (OLO) is a problem where an algorithm repeatedly
chooses a point $w_t$ from a convex decision set $K$, observes an arbitrary, or
even adversarially chosen, loss vector $\ell_t$ and suffers loss $\langle
\ell_t, w_t \rangle$.  The goal of the algorithm is to have a small cumulative
loss. Performance of an algorithm is evaluated by the so-called regret, which
is the difference of cumulative losses of the algorithm and of the
(hypothetical) strategy that would choose in every round the same best point in
hindsight.

OLO is a fundamental problem in machine
learning~\cite{Cesa-Bianchi-Lugosi-2006, Rakhlin-Sridharan-2009,
Shalev-Shwartz-2011}.  Many learning problems can be directly phrased as OLO,
e.g., learning with expert advice~\cite{Littlestone-Warmuth-1994, Vovk-1998,
Freund-Schapire-1997, Cesa-Bianchi-Haussler-Helmbold-Schapire-Warmuth-1997},
online combinatorial optimization~\cite{Kalai-Vempala-2005,
Helmbold-Warmuth-2009, Koolen-Warmuth-Kivinen-2010}. Other problems can be
reduced to OLO, e.g. online convex
optimization~\cite{Abernethy-Bartlett-Rakhlin-Tewari-2008},
\cite[Chapter~2]{Shalev-Shwartz-2011}, online classification
~\cite{Rosenblatt-1958,Freund-Schapire-1999} and
regression~\cite{Kivinen-Warmuth-1997},
~\cite[Chapters~11~and~12]{Cesa-Bianchi-Lugosi-2006}, multi-armed
problems~\cite[Chapter~6]{Cesa-Bianchi-Lugosi-2006},
\cite{Abernethy-Hazan-Rakhlin-2008, Bubeck-Cesa-Bianchi-2012}, and batch
and stochastic optimization of convex functions~\cite{Nemirovski-Yudin-1983,
Bubeck-2015}.  Hence, a result in OLO immediately implies other results in all
these domains.

The adversarial choice of the loss vectors received by the algorithm is what
makes the OLO problem challenging. In particular, if an OLO algorithm commits
to an upper bound on the norm of future loss vectors, its regret can be made
arbitrarily large through an adversarial strategy that produces loss vectors
with norms that exceed the upper bound.

For this reason, most of the existing OLO algorithms receive as an input---or
explicitly assume---an upper bound $B$ on the norm of the loss vectors.  The
input $B$ is often disguised as the learning rate, the regularization
parameter, or the parameter of strong convexity of the regularizer. Examples of
such algorithms include the \textsc{Hedge} algorithm or online projected
gradient descent with fixed learning rate.  However, these algorithms have two
obvious drawbacks.

First, they do not come with any regret guarantee for sequences of loss vectors
with norms exceeding $B$. Second, on sequences where the norm of loss vectors
is bounded by $b \ll B$, these algorithms fail to have an optimal regret
guarantee that depends on $b$ rather than on $B$.

% Change spacing between rows.
\renewcommand{\arraystretch}{1.8}

\begin{table}[t]
\fontsize{8}{8.2}\selectfont
\centering
\begin{tabular}{|p{3.6cm}|c|p{3.4cm}|c|}
\hline
\textbf{Algorithm} & \textbf{Decisions Set(s)} & \textbf{Regularizer(s)} & \textbf{Scale-Free} \\ \hline \hline
\textsc{Hedge} \cite{Freund-Schapire-1997} & Probability Simplex & Negative Entropy & No \\ \hline
\textsc{GIGA} \cite{Zinkevich-2003} & Any Bounded & $\frac{1}{2}\|w\|_2^2$ & No \\ \hline
\textsc{RDA} \cite{Xiao-2010} & \textbf{Any} & \textbf{Any Strongly Convex} & No \\ \hline
\textsc{FTRL-Proximal} \cite{McMahan-Streeter-2010,McMahan-2014} & Any Bounded & $\frac{1}{2}\|w\|_2^2 + $ any convex func. & \textbf{Yes} \\ \hline
\textsc{AdaGrad MD} \cite{Duchi-Hazan-Singer-2011} & Any Bounded & $\frac{1}{2}\|w\|_2^2 + $ any convex func. & \textbf{Yes} \\ \hline
\textsc{AdaGrad FTRL} \cite{Duchi-Hazan-Singer-2011} & \textbf{Any} & $\frac{1}{2}\|w\|_2^2 + $ any convex func. & No \\ \hline
\textsc{AdaHedge} \cite{de-Rooij-van-Erven-Grunwald-Koolen-2014} & Probability Simplex & Negative Entropy & \textbf{Yes} \\ \hline
\textsc{Optimistic MD} \cite{Rakhlin-Sridharan-2013} & $\sup_{u,v \in K} \Breg_f(u,v) < \infty$ & \textbf{Any Strongly Convex} & \textbf{Yes} \\ \hline
\textsc{NAG} \cite{Ross-Mineiro-Langford-2013} & $\{u: \max_t \langle \ell_t, u\rangle \le C\}$ & $\frac{1}{2}\|w\|_2^2 $& Partially\footnotemark \\ \hline
\textsc{Scale invariant algorithms} \cite{Orabona-Crammer-Cesa-Bianchi-2014} & \textbf{Any} & $\frac{1}{2}\|w\|_p^2 + $ any convex func. \newline $1 < p \le 2$ & Partially\textsuperscript{\ref{footnote-label}} \\ \hline
\textsc{AdaFTRL} \textbf{[this paper]} & Any Bounded & \textbf{Any Strongly Convex} & \textbf{Yes} \\ \hline
\textsc{SOLO FTRL} \textbf{[this paper]} & \textbf{Any} & \textbf{Any Strongly Convex} & \textbf{Yes} \\ \hline
\end{tabular}
\caption{Selected results for OLO. Best results in each column are in bold.
\label{table:results}
}
\end{table}

\footnotetext{\label{footnote-label} These algorithms attempt to produce an
invariant sequence of predictions $\langle w_t, \ell_t \rangle$, rather than a
sequence of invariant $w_t$.}

There is a clear practical need to design algorithms that adapt automatically
to norms of the loss vectors.  A natural, yet overlooked, design method to
achieve this type of adaptivity is by insisting to have a \textbf{scale-free}
algorithm.  That is, the sequence of decisions of the algorithm does not change
if the sequence of loss vectors is multiplied by a positive constant.

A summary of algorithms for OLO is presented in Table~\ref{table:results}.
While the scale-free property has been looked at in the expert setting, in the
general OLO setting this issue has been largely ignored.  In particular, the
\textsc{AdaHedge}~\cite{de-Rooij-van-Erven-Grunwald-Koolen-2014} algorithm, for
prediction with expert advice, is specifically designed to be scale-free.  A
notable exception in the OLO literature is the discussion of the ``off-by-one''
issue in~\cite{McMahan-2014}, where it is explained that even the popular
\textsc{AdaGrad} algorithm~\cite{Duchi-Hazan-Singer-2011} is not completely
adaptive; see also our discussion in Section~\ref{section:solo-ftrl}. In
particular, existing scale-free algorithms cover only some norms/regularizers
and \emph{only} bounded decision sets. The case of \textbf{unbounded decision
sets}, practically the most interesting one for machine learning applications,
remains completely unsolved.

Rather than trying to design strategies for a particular form of loss vectors
and/or decision sets, in this paper we explicitly focus on the scale-free
property. Regret of scale-free algorithms is proportional to the scale of the
losses, ensuring optimal linear dependency on the maximum norm of the loss
vectors.

We design and analyze three scale-free algorithms. First two algorithms, which
we call \textsc{AdaFTRL} and \textsc{SOLO FTRL}, are instances of
\textsc{Follow The Regularized Leader} (\textsc{FTRL}) with adaptive learning
rate.  \textsc{AdaFTRL} is a generalization of
\textsc{AdaHedge}~\cite{de-Rooij-van-Erven-Grunwald-Koolen-2014} to arbitrary
strongly convex regularizers.  \textsc{SOLO FTRL} can be viewed as the
``correct'' scale-free version of \textsc{AdaGrad
FTRL}~\cite{Duchi-Hazan-Singer-2011} generalized to arbitrary strongly convex
regularizers.  The third algorithm, which we call \textsc{Scale-Free Mirror
Descent}, is a variant of \textsc{Mirror Descent}. It is a generalization of
\textsc{AdaGrad MD}~\cite{Duchi-Hazan-Singer-2011} to arbitrary strongly convex
regularizers, similar to \textsc{Optimistic MD}~\cite{Rakhlin-Sridharan-2013}.
The three algorithms are presented and analyzed in
Sections~\ref{section:ada-ftrl}, \ref{section:solo-ftrl} and
\ref{section:mirror-descent} respectively.

For all three algorithms, we prove that for bounded decision sets the regret
after $T$ rounds is at most $O \left(\sqrt{\sum_{t=1}^T\|\ell_t\|_*^2} \right)$
where the constant hidden in $O(\cdot)$ depends on the decision set and the
regularizer.  In Section~\ref{section:lower-bound}, we show that the
$\sqrt{\sum_{t=1}^T \|\ell_t\|_*^2}$ term is necessary by proving $\Omega
\left(D \sqrt{\sum_{t=1}^T\|\ell_t\|_*^2} \right)$ lower bound on the regret of
any algorithm for OLO for any decision set with diameter $D$ with respect to
the primal norm $\|\cdot\|$.

For the \textsc{SOLO FTRL} algorithm, we prove an $O \left(\max_{t=1,2,\dots,T}
\|\ell_t\|_* \sqrt{T} \right)$ regret bound for any unbounded decision set. The
constant hidden in $O(\cdot)$ depends not only on the decision set and the
regularizer, but also on the competitor $u \in K$.~\footnote{An upper bound
that does not depend on $u$ is impossible. It would contradict the $\Omega
\left( D \sqrt{\sum_{t=1}^T\|\ell_t\|_*^2} \right)$ lower bound.} This is the
\textbf{first adaptive algorithm for unbounded decision sets} with a
non-trivial upper bound.

All three algorithms are \textbf{any-time}, i.e., they do not need to know the
number of rounds in advance and the regret bounds hold for all time steps
simultaneously.

Our proofs of the regret upper bounds rely on homogeneous inequalities
(Lemmas~\ref{lemma:recurrence-solution},
\ref{lemma:sum-of-square-roots-inverses}, \ref{lemma:useful}).  Two of the
inequalities are new. Previously, many proofs of regret bounds relied on ad-hoc
non-homogeneous inequalities, or the existing algorithms used quantities (e.g.
$\sum_{t=1}^T \|\ell_t\|_*^2$) that are not available until all loss vectors
are processed.

Finally, in Section~\ref{subsection:mirror-descent-lower-bound}, we show
negative results for existing popular variants of \textsc{Mirror Descent}. We
show that variants of \textsc{Mirror Descent} have $\Omega(T)$ regret if the
Bregman divergence associated with the regularizer is unbounded on the decision
set.  The setting includes the standard gradient descent with step size
$1/\sqrt{t}$, \textsc{AdaGrad MD} and \textsc{Scale-Free Mirror Descent} on a
unbounded decision set. This result suggests that \textsc{Follow The
Regularized Leader} is superior to \textsc{Mirror Descent}.
