\section{Limits}
\label{section:limits}

In this section we show that prediction of \textsc{AdaFTRL} is correctly
defined when the regularizer is multiplied by zero.

\begin{lemma}[Prediction for Zero Regularizer]
\label{lemma:prediction-limit-existence}
Let $K$ be non-empty bounded closed convex subset of a finite dimensional
normed real vector space $(V, \|\cdot\|)$.  Let $R:K \to \R$ be strictly convex
and lower semi-continuous, and let $L \in V^*$. The limit
\begin{equation}
\label{equation:limit}
\lim_{\eta \to +\infty}
\argmin_{w \in K} \left( \langle L, w \rangle + \frac{1}{\eta} R(w) \right)
\end{equation}
exists and it is equal to the unique minimizer of $R(w)$ over the set (of minimizers)
$$
\left\{ w \in K ~:~ \langle L, w \rangle = \inf_{v \in K} \langle L, v \rangle \right\} \; .
$$
\end{lemma}

Before we give the proof, we illustrate the lemma on a simple
example.  Let $K = [-1,1]^2$ be a closed square in $\R^2$ and let
$R(w) = \|w\|_2^2$. Let $L = (1,0)$. The minimizers are
$$
\argmin_{w \in K} \langle L, w \rangle = \{ (-1,y) ~:~ y \in [-1,1] \} \; .
$$
The minimizer with the smallest value of $R(w)$ is $(-1,0)$. Hence the lemma
implies that
$$
\lim_{\eta \to +\infty} \argmin_{w \in K}
\left( \langle L, w \rangle + \frac{1}{\eta} \|w\|_2^2 \right) = (-1, 0) \; .
$$

\begin{proof}[Proof of Lemma~\ref{lemma:prediction-limit-existence}]
Without loss of generality, we can assume that $R(w)$ is non-negative for any
$w \in K$.  For otherwise, we can replace $R(w)$ with $R'(w) = R(w) - \inf_{v
\in K} R(v)$.

Since $K$ is a non-empty bounded closed convex subset of a finite dimensional
normed vector space, it is compact and $r^* = \min_{w \in K} \langle L, w
\rangle$ exists and is attained at some $w \in K$. Consider the hyperplane
$$
H = \{ w \in V ~:~ \langle L, w \rangle = r^* \} \; .
$$
The intersection $H \cap K$ is a non-empty compact convex set.
Let
$$
v^* = \argmin_{v \in K \cap H} R(v) \; .
$$
The existence of $v^*$ follows from compactness of $H \cap K$ and lower
semi-continuity of $R(v)$.  Uniqueness of $v^*$ follows from strict convexity
of $R(v)$. We show that the limit (\ref{equation:limit}) equals $v^*$.

By the definition of $H$,
\begin{equation}
\label{equation:limit-is-minimizer}
v^* \in \argmin_{w \in K} \ \langle L, w \rangle \; .
\end{equation}
Let $S = \{ w \in K ~:~ R(w) \le R(v^*) \}$. Since $R(w)$ is lower
semi-continuous $S$ is closed. Since $R(w)$ is strictly convex, $S \cap H =
\{v^*\}$.

For any $\eta > 0$, let
$$
w(\eta) = \argmin_{w \in K} \left( \langle L, w \rangle + \frac{1}{\eta} R(w) \rangle \right) \; .
$$
We prove that $w(\eta) \in S$. Indeed, by optimality of $v^*$ and $w(\eta)$
$$
\frac{1}{\eta} R(w(\eta))  + \langle L, w(\eta) \rangle
\le
\frac{1}{\eta} R(v^*)  + \langle L, v^* \rangle
\le
\frac{1}{\eta} R(v^*)  + \langle L, w(\eta) \rangle \; .
$$
and hence $R(w(\eta)) \le R(v^*)$.

By non-negativity of $R$ and optimality of $w(\eta)$ we have
$$
\langle L, w(\eta) \rangle
\le \langle L, w(\eta) \rangle + \frac{1}{\eta} R(w(\eta))
\le \langle L, v^* \rangle + \frac{1}{\eta} R(v^*) \; .
$$
Taking the limit $\eta \to +\infty$, we see that
$$
\lim_{\eta \to +\infty} \langle L, w(\eta) \rangle
\le\lim_{\eta \to +\infty} \left( \langle L, v^* \rangle + \frac{1}{\eta} R(v^*) \right)
= \langle L, v^* \rangle \; .
$$
From~(\ref{equation:limit-is-minimizer}) we have
$\langle L, v^* \rangle \le \langle L, w \rangle$ for any $w$, and therefore
\begin{equation}
\label{equation:limit-equality}
\lim_{\eta \to +\infty} \langle L, w(\eta) \rangle = \langle L, v^* \rangle \; .
\end{equation}

Consider any sequence $\{\eta_t\}_{t=1}^\infty$ of positive numbers approaching
$+\infty$.  Since $K$ is compact, $w(\eta_t)$ has a convergent subsequence.
Thus $\{w(\eta_t)\}_{t=1}^\infty$ has at least one accumulation point; let
$w^*$ be any of them. We will show that $w^* = v^*$.

Consider a subsequence $\{\xi_t\}_{t=1}^\infty$ of $\{\eta_t\}_{t=1}^\infty$
such that $\lim_{t \to \infty} w(\xi_t) = w^*$.  Since $w(\xi_t) \in S$ and $S$
is closed, $w^* \in S$.  From (\ref{equation:limit-equality}) we have $\langle
L, w^* \rangle = \langle L, v^* \rangle$ and hence $w^* \in H$. Thus $w^* \in S
\cap H$.  Since $v^*$ is the only point in $S \cap H$ we must have $w^* = v^*$.
\end{proof}

\begin{lemma}[Limit of Bregman Divergence]
\label{lemma:limit-bregman-divergence}
Let $K$ be a non-empty bounded closed convex subset of a finite dimensional
normed real vector space $(V, \|\cdot\|)$.  Let $R:K \to \R$ be a strongly
convex lower semi-continuous function bounded from above. Then, for any $x,y
\in V^*$,
$$
\lim_{a \to 0^+} a \Breg_{R^*}(x/a, y/a) = \langle x, u - v \rangle \\
$$
where
\begin{align*}
u = \lim_{a \to 0^+} \argmin_{w \in K} \left( a R(w) - \langle x, w \rangle \right)
\quad \text{and} \quad
v = \lim_{a \to 0^+} \argmin_{w \in K} \left( a R(w) - \langle y, w \rangle \right) \; .
\end{align*}
\end{lemma}

\begin{proof}
Using property 3 of Proposition~\ref{proposition:conjugate-properties} we can write
the divergence
\begin{align*}
a \Breg_{R^*}(x/a, y/a) & = a R^*(x/a) - a R^*(y/a) - \langle x - y, \grad R^*(y/a) \rangle \\
& =
 a \left[ \langle x/a, \grad R^*(x/a) \rangle - R(\grad R^*(x/a)) \right] \\
& \qquad - a \left[ \langle y/a, \grad R^*(y/a) \rangle - R(\grad R^*(y/a)) \right]
- \langle x - y, \grad R^*(y/a) \rangle \\
& =
\langle x, \grad R^*(x/a) - \grad R^*(y/a) \rangle - a R(\grad R^*(x/a))
+ a R(\grad R^*(y/a)) \; .
\end{align*}
Property 2 of Proposition~\ref{proposition:conjugate-properties} implies that
\begin{align*}
u = \lim_{a \to 0^+} \grad R^*(x/a) & = \lim_{a \to 0^+} \argmin_{w \in K} \left( a R(w) - \langle x, w \rangle \right) \; , \\
v = \lim_{a \to 0^+} \grad R^*(y/a) & = \lim_{a \to 0^+} \argmin_{w \in K} \left( a R(w) - \langle y, w \rangle \right) \; .
\end{align*}
The limits on the right exist because of compactness of $K$. They are simply
the minimizers $u= \argmin_{w \in K} - \langle x, w \rangle$ and $v= \argmin_{w
\in K} - \langle y, w \rangle$ where ties in $\argmin$ are broken according to
smaller value of $R(w)$.

By assumption $R(w)$ is upper bounded. It is also lower bounded, since it is
defined on a compact set and it is lower semi-continuous. Thus,
\begin{align*}
& \lim_{a \to 0^+} a \Breg_{R^*}(x/a, y/a) \\
& = \lim_{a \to 0^+} \langle x, \grad R^*(x/a) - \grad R^*(y/a) \rangle - a R(\grad R^*(x/a)) + a R(\grad R^*(y/a)) \\
& = \lim_{a \to 0^+} \langle x, \grad R^*(x/a) - \grad R^*(y/a) \rangle = \langle x, u - v \rangle \; .
\end{align*}
\end{proof}
