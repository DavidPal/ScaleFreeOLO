\section{Conclusions}
\label{section:conclusions}

We have investigated scale-free algorithms for online linear optimization and
we have shown that the scale-free property leads to algorithms which have optimal
regret and do not need to know or assume \textbf{anything} about the sequence
of loss vectors. In particular, the algorithms do not assume any upper or lower
bounds on the norms of the loss vectors or the number of rounds.

We have designed a scale-free algorithm based on \textsc{Follow The Regularizer
Leader}. Its regret with respect to any competitor $u$ is
$$
O \left(f(u) \sqrt{\sum_{t=1}^T \|\ell_t\|_*^2}
+ \min\{\sqrt{T}, D\} \max_{t=1,2,\dots,T} \|\ell_t\|_* \right) \; ,
$$
where $f$ is any non-negative $1$-strongly convex
function defined on the decision set and $D$ is the diameter of the decision
set. The result makes sense even when the decision set is unbounded.

A similar, but weaker result holds for a scale-free algorithm based on
\textsc{Mirror Descent}. However, we have also shown that this algorithm is
strictly weaker than algorithms based on \textsc{Follow The Regularizer Leader}.
Namely, we gave examples of regularizers for which the scale-free version of
\textsc{Mirror Descent} has $\Omega(T)$ regret or worse.

We have proved an $\frac{D}{\sqrt{8}} \sqrt{\sum_{t=1}^T \|\ell\|_*^2}$ lower
bound on the regret of any algorithm for any decision set with diameter $D$.

Notice that with the regularizer $f(u) = \frac{1}{2}\|u\|_2^2$ the regret of
\textsc{SOLO FTRL} depends quadratically on the norm of the competitor
$\|u\|_2$. There exist non-scale-free algorithms \cite{McMahan-Streeter-2012,
McMahan-Abernethy-2013, Orabona-2013, McMahan-Orabona-2014, Orabona-2014,
Orabona-Pal-2016} that have only a $O(\|u\|_2 \sqrt{\log \|u\|_2})$ or
$O(\|u\|_2 \log \|u\|_2)$ dependency.  These algorithms assume an a priori bound
on the norm of the loss vectors. Recently, an algorithm that adapts to norms of
loss vectors and has $O(\|u\|_2 \log \|u\|_2)$ dependency was
proposed~\cite{Cutkosky-Boahen-2016}. However, the trade-off between the
dependency on $\norm{u}_2$ and the adaptivity to the norms of the loss vectors
still remains to be explored.
