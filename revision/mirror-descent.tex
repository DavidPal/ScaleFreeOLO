\section{Scale-Free Mirror Descent}
\label{section:mirror-descent}

In this section, we analyze scale-free version of \textsc{Mirror Descent}.
Our algorithm uses the regularizer
%
\begin{equation}
\label{equation:scale-free-mirror-descent}
R_t(w) = R(w) \sqrt{\sum_{i=1}^t \|\ell_i\|_*^2} \; ,
\end{equation}
%
where $R:K \to \R$ an arbitrary strongly convex function.  As for \textsc{SOLO
FTRL}, it is easy to see that such regularizer gives rise to predictions $w_t$
that are scale-free.  We call the resulting algorithm \textsc{Scale-Free MD}.
Similar to \textsc{SOLO FTRL}, the regularizer
\eqref{equation:scale-free-mirror-descent} does not uniquely define the
\textsc{MD} minimizer $w_{t+1} = \argmin_{w \in K} \left(\langle \ell_t, w
\rangle + \Breg_{R_t}(w, w_t) \right)$ when $\sqrt{\sum_{i=1}^t
\|\ell_i\|_*^2}$ is zero.  This happens when the loss vectors $\ell_1, \ell_2,
\dots, \ell_t$ are all zero. In this case, we define $w_{t+1} = \argmin_{w \in
K} R(w)$ which agrees with $w_{t+1} = \lim_{a \to 0^+} \argmin_{w \in K} a
\Breg_{R}(w, w_t)$.  Similarly, $w_1 = \argmin_{w \in K} R(w)$.

Per-coordinate version of \textsc{Scale-Free MD} with regularizer $R(w) =
\frac{1}{2}\norm{w}_2^2$ is exactly the same algorithm as the diagonal version
of \textsc{AdaGrad MD}~\cite{Duchi-Hazan-Singer-2011}.

The theorem below upper bounds the regret of \textsc{Scale-Free MD} (see also
\cite{Duchi-Hazan-Singer-2011, Duchi-Shalev-Shwartz-Singer-Tewari-2010,
Rakhlin-Sridharan-2013}).  The proof is in~\ref{section:mirror-descent-proofs}.

\begin{theorem}[Regret of Scale-Free Mirror Descent]
\label{theorem:regret-scale-free-mirror-descent}
Suppose $K \subseteq V$ is a non-empty closed convex set. Suppose that $R:K
\to \R$ is a $\lambda$-strongly convex function with respect to a norm
$\|\cdot\|$.  \textsc{Scale-Free MD} with regularizer $R$ satisfies
for any $u \in K$,
$$
\Regret_T(u)
\le
\left( \frac{1}{\lambda} + \sup_{v \in K} \Breg_R(u,v) \right) \sqrt{\sum_{t=1}^T \|\ell_t\|_*^2} \; .
$$
\end{theorem}

We choose the regularizer $R(w) = \lambda f(w)$ where $f$ is a $1$-strongly convex
function and optimize $\lambda$. The result is the following Corollary. Its
proof is trivial.

\begin{corollary}[Regret of Scale-Free Mirror Descent]
\label{corollary:regret-scale-free-mirror-descent}
Suppose $K \subseteq V$ is a non-empty bounded closed convex set.  Suppose
that $f:K \to \R$ is a $1$-strongly convex function with respect to a norm
$\|\cdot\|$.  \textsc{Scale-Free MD} with regularizer
$$
R(w) = \frac{f(w)}{\displaystyle \sqrt{\sup_{u,v \in K} \Breg_f(u,v)}}
\quad \text{satisfies} \quad %
\Regret_T \le 2 \sqrt{\sup_{u,v \in K} \Breg_f(u,v) \sum_{t=1}^T \|\ell_t\|_*^2} \; .
$$
\end{corollary}

The regret bound for \textsc{Scale-Free MD} in the
Corollary~\ref{corollary:regret-scale-free-mirror-descent} depends on
$\sup_{u,v \in K} \Breg_f(u,v)$. In contrast, the regret bound for
\textsc{SOLO FTRL} in Corollary~\ref{corollary:regret-solo-ftrl-bounded-set}
depend on $\sup_{u \in K} f(u)$.  Similarly, the regret bound in
Theorem~\ref{theorem:regret-scale-free-mirror-descent} for \textsc{Scale-Free
MD} depends on $\sup_{v \in K} \Breg_R(u,v)$ and the regret bounds in
Theorem~\ref{theorem:regret-solo-ftrl}
for \textsc{SOLO FTRL} depend on $R(u)$. It is not hard to
show that
\begin{equation}
\label{equation:regularizer-vs-divergence}
\forall u \in K \qquad R(u) \le \sup_{v \in K} \Breg_R(u,v) \; ,
\end{equation}
provided that at the minimizer $v^* = \argmin_{v \in K} R(v)$ both $R(v^*)$ and
$\grad R(v^*)$ are zero. Indeed, in that case, $R(u) = \Breg_R(u,v^*) \le
\sup_{v \in K} \Breg_R(u,v)$.

The assumption $R(v^*) = 0$ and $\grad R(v^*) = 0$
are easy to achieve by adding an affine function to the regularizer:
$$
R'(u) = R(u) - \langle \grad R(v^*), u - v^* \rangle - R(v^*) \; .
$$
The regularizer $R'$ has the same parameter of strong convexity as $R$, the
associated Bregman divergences $\Breg_{R'}$ and $\Breg_{R}$ are equal, $R'$ and
$R$ have the same minimizer $v^*$, and $R'(v^*)$ and $\grad R'(v^*)$ are both
zero.

Thus, inequality \eqref{equation:regularizer-vs-divergence} implies
that---ignoring constant factors---the regret bound for \textsc{Scale-Free MD}
is inferior to the regret bound for \textsc{SOLO FTRL}.
In fact, it is not hard to come up with examples where $R(u)$ is finite whereas
$\sup_{v \in K} \Breg_R(u,v)$ is infinite. We mention two such examples. The
first example is $R(w) = \frac{1}{2}\|w\|_2^2$ defined on the whole space $V$,
where for any $u \in V$, $R(u)$ is a finite value but $\sup_{v \in K}
\Breg_R(u,v) = \sup_{v \in V} \frac{1}{2}\|u-v\|_2^2 = +\infty$. The second
example is the shifted negative entropy regularizer $R(w) = \ln(d) +
\sum_{j=1}^d w_j \ln w_j$ defined on the $d$-dimensional probability simplex $K
= \{ w \in \R^d ~:~ w_j \ge 0, \sum_{j=1}^d w_j = 1 \}$, where for any $u \in
K$, $R(u)$ is finite and in fact lies in the interval $[0, \ln d]$ but
$\sup_{v \in K} \Breg_R(u,v) = \sup_{v \in K} \sum_{j=1}^d u_j \ln(u_j/v_j) = +
\infty$. We revisit these examples in the following subsection.


\subsection{Lower Bounds for Scale-Free Mirror Descent}
\label{subsection:mirror-descent-lower-bound}

The bounds in Theorem~\ref{theorem:regret-scale-free-mirror-descent} and
Corollary~\ref{corollary:regret-scale-free-mirror-descent} are vacuous when
$\Breg_R(u,v)$ is not bounded. One might wonder if the assumption that
$\Breg_R(u,v)$ is bounded is necessary in order for \textsc{Scale-Free MD} to
have a sublinear regret. We show necessity of this assumption on two
counter-examples.  In these counter-examples, we consider strongly convex
regularizers $R$ such that $\Breg_R(u,v)$ is not bounded and we construct
sequences of loss vectors $\ell_1, \ell_2, \dots, \ell_T$ such that
$\|\ell_1\|_* = \|\ell_2\|_* = \dots = \|\ell_T\|_* = 1$ and \textsc{Scale-Free
MD} has regret $\Omega(T)$ or worse.

The first counter-example is stated as
Theorem~\ref{theorem:first-counter-example} below; our proof is
in~\ref{section:mirror-descent-proofs}. The decision set is the whole space
$K=V$ and the regularizer is $R(w) = \frac{1}{2}\|w\|_2^2$. Note that $R(w)$ is
$1$-strongly convex with respect to $\|\cdot\|_2$ and the dual norm of
$\|\cdot\|_2$ is $\|\cdot\|_2$. The corresponding Bregman divergence is
$\Breg_R(u,v) = \frac{1}{2}\|u-v\|_2^2$. The counter-example constructs a
sequence of unit-norm loss vectors in the one-dimensional subspace spanned by
the first vector of the standard orthnormal basis.  On such a sequence, both
versions of \textsc{AdaGrad MD} as well as  \textsc{Scale-Free MD} are identical
to gradient descent with step size $1/\sqrt{t}$, i.e., they are identical
Zinkevich's \textsc{Generalized Infinitesimal Gradient Ascent} (\textsc{GIGA})
algorithm~\cite{Zinkevich-2003}. Hence the lower bound applies to all these
algorithms.

\begin{theorem}[First Counter-Example]
\label{theorem:first-counter-example}
Suppose $K = V$. For any $T \ge 42$, there exists a sequence of loss vectors
$\ell_1, \ell_2, \dots, \ell_T \in V^*$ such that $\|\ell_1\|_2 = \|\ell_2\|_2
= \dots = \|\ell_T\|_2 = 1$ and \textsc{Scale-Free MD} with
regularizer $R(w) = \frac{1}{2}\|w\|_2^2$, \textsc{GIGA},
and both versions of \textsc{AdaGrad MD} satisfy
$$
\Regret_T(0) \ge \frac{T^{3/2}}{20} \; .
$$
\end{theorem}

The second counter-example is stated as
Theorem~\ref{theorem:second-counter-example} below; our proof is
in~\ref{section:mirror-descent-proofs}.  The decision set is the
$d$-dimensional probability simplex $K = \{ w \in \R^d ~:~ w_j \ge 0,
\sum_{j=1}^d w_j = 1 \}$ and the regularizer is the negative entropy $R(w) =
\sum_{j=1}^d w_j \ln w_j$.  Negative entropy is $1$-strongly convex with
respect to $\|\cdot\|_1$ and the dual norm of $\|\cdot\|_1$ is
\mbox{$\|\cdot\|_\infty$}.  The corresponding Bregman divergence is the
Kullback-Leibler divergence $\Breg_R(u,v) = \sum_{j=1}^d u_j \ln(u_j/v_j)$.
Note that despite that negative entropy is upper- and lower-bounded,
Kullback-Leibler divergence can be arbitrarily large.

\begin{theorem}[Second Counter-Example]
\label{theorem:second-counter-example}
Let $d \ge 2$, let $V = \R^d$, and let $K = \{ w \in V ~:~ w_j \ge 0,
\sum_{j=1}^d w_j = 1 \}$ be the $d$-dimensional probability simplex.  For any
$T \ge 120$, there exists a sequence of loss vectors $\ell_1, \ell_2, \dots,
\ell_T \in V^*$ such that $\|\ell_1\|_\infty = \|\ell_2\|_\infty = \dots =
\|\ell_T\|_\infty = 1$ and \textsc{Scale-Free MD} with regularizer
$R(w) = \sum_{j=1}^d w_j \ln w_j$ satisfies
$$
\Regret_T \ge \frac{T}{6} \; .
$$
\end{theorem}
