\section{\textsc{SOLO FTRL}}
\label{section:solo-ftrl}

In this section, we introduce our first scale-free algorithm; it will be based
on \textsc{FTRL}. The closest algorithm to a scale-free \textsc{FTRL} in the
existing literature is the \textsc{AdaGrad FTRL}
algorithm~\cite{Duchi-Hazan-Singer-2011}. It uses a regularizer on each
coordinate of the form
\begin{equation*}
R_t(w) = R(w) \left(\delta + \sqrt{\sum_{i=1}^{t-1} \|\ell_i\|_*^2} \right).
\end{equation*}
This kind of regularizer would yield a scale-free algorithm \emph{only} for
$\delta=0$.  Unfortunately, the regret bound in~\cite{Duchi-Hazan-Singer-2011}
becomes vacuous for such setting in the unbounded case. In fact, it requires
$\delta$ to be greater than $\|\ell_t\|_*$ for all time steps $t$, requiring
knowledge of the future (see Theorem~5 in~\cite{Duchi-Hazan-Singer-2011}). In
other words, despite of its name, \textsc{AdaGrad FTRL} is not fully adaptive to
the norm of the gradient vectors. Similar considerations hold for the
\textsc{FTRL-Proximal} in \cite{McMahan-Streeter-2010,McMahan-2014}: The
scale-free setting of the learning rate is valid only in the bounded case.

One simple approach would be to use a doubling trick on $\delta$ in order to
estimate on the fly the maximum norm of the losses. Note that a naive strategy
would still fail because the initial value of $\delta$ should be data-dependent
in order to have a scale-free algorithm. Moreover, we would have to upper bound
the regret in all the rounds where the norm of the current loss is bigger than
the estimate. Finally, the algorithm would depend on an additional parameter,
the ``doubling'' power. Hence, even in the case one would prove a regret bound,
such strategy would give the feeling that \textsc{FTRL} needs to be ``fixed''
in order to obtain a scale-free algorithm.

In the following, we propose a much simpler and better approach.  We propose to
use Algorithm~\ref{algorithm:ftrl-varying-regularizer} with the regularizer
\begin{equation}
\label{equation:solo-ftrl-regularizer}
R_t(w) = R(w) \sqrt{\sum_{i=1}^{t-1} \|\ell_i\|_*^2} \; ,
\end{equation}
where $R:K \to \R$ is any strongly convex function. Through a refined analysis,
we show that this regularizer suffices to obtain an optimal regret bound for any
decision set, bounded or unbounded.  We call this variant \textsc{Scale-free
Online Linear Optimization FTRL} algorithm (\textsc{SOLO FTRL}).  Our main
result is Theorem~\ref{theorem:regret-solo-ftrl} below, which is proven in
Section~\ref{section:solo-ftrl-regret-bound}.

The regularizer~\eqref{equation:solo-ftrl-regularizer} does not uniquely define
the \textsc{FTRL} minimizer $w_t = \argmin_{w \in K} R_t(w)$ when
$\sqrt{\sum_{i=1}^{t-1} \|\ell_i\|_*^2}$ is zero. This happens if $\ell_1,
\ell_2, \dots, \ell_{t-1}$ are all zero (and in particular for $t=1$).  In that
case, we define $w_t = \argmin_{w \in K} R(w)$ which is consistent with
$w_t = \lim_{a \to 0^+}  \argmin_{w \in K} a R(w)$.

\begin{theorem}[Regret of \textsc{SOLO FTRL}]
\label{theorem:regret-solo-ftrl}
Suppose $K \subseteq V$ is a non-empty closed convex set.  Let $D = \sup_{u,v
\in K} \|u - v\|$ be its diameter with respect to a norm $\|\cdot\|$.  Suppose
that the regularizer $R:K \to \R$ is a non-negative lower semi-continuous
function that is $\lambda$-strongly convex with respect to $\|\cdot\|$. The
regret of \textsc{SOLO FTRL} satisfies
\begin{align*}
\Regret_T(u)
& \le \left( R(u) + \frac{2.75}{\lambda}\right) \sqrt{\sum_{t=1}^{T} \norm{\ell_t}_*^2}
+ 3.5 \min\left\{\frac{\sqrt{T-1}}{\lambda} , D\right\} \max_{t \le T} \|\ell_t\|_* \; .
\end{align*}
\end{theorem}

When $K$ is unbounded, we pay a penalty that scales as $\max_{t \le T}
\|\ell_t\|_* \sqrt{T}$, that has the same magnitude of the first term in the
bound. On the other hand, when $K$ is bounded, the second term is a constant and
we can choose the optimal multiple of the regularizer.  We choose $R(w) =
\lambda f(w)$ where $f$ is a $1$-strongly convex function and optimize
$\lambda$.  The result of the optimization is
Corollary~\ref{corollary:regret-solo-ftrl-bounded-set}.

\begin{corollary}[Regret Bound for Bounded Decision Sets]
\label{corollary:regret-solo-ftrl-bounded-set}
Suppose $K \subseteq V$ is a non-empty bounded closed convex set.  Suppose that
$f:K \to \R$ is a non-negative lower semi-continuous function that is
$1$-strongly convex with respect to $\|\cdot\|$. \textsc{SOLO FTRL} with
regularizer
$$
R(w) = \frac{f(w)\sqrt{2.75}}{\sqrt{\sup_{v \in K} f(v)}}
\quad \text{satisfies} \quad
\Regret_T \le 13.3 \sqrt{\sup_{v \in K} f(v) \sum_{t=1}^{T} \norm{\ell_t}_*^2} \; .
$$
\end{corollary}
%
\begin{proof}
Let $S = \sup_{v \in K} f(v)$. Theorem~\ref{theorem:regret-solo-ftrl} applied
to the regularizer $R(w) = \frac{c}{\sqrt{S}} f(w)$, together with
Proposition~\ref{proposition:diameter-vs-range} and a crude bound
$\max_{t=1,2,\dots,T} \|\ell_t\|_* \le \sqrt{\sum_{t=1}^T \|\ell_t\|_*^2}$,
give
$$
\Regret_T \le \left(c + \frac{2.75}{c}  + 3.5\sqrt{8} \right) \sqrt{S \sum_{t=1}^{T} \norm{\ell_t}_*^2} \; .
$$
We choose $c$ by minimizing $g(c) = c + \frac{2.75}{c} + 3.5\sqrt{8}$. Clearly,
$g(c)$ has minimum at $c = \sqrt{2.75}$ and has minimal value $g(\sqrt{2.75}) =
2\sqrt{2.75} + 3.5\sqrt{8} \le 13.3$.
\end{proof}


\subsection{Proof of Regret Bound for \textsc{SOLO FTRL}}
\label{section:solo-ftrl-regret-bound}

The proof of Theorem~\ref{theorem:regret-solo-ftrl} relies on an inequality
(Lemma~\ref{lemma:useful}).  Related and weaker inequalities, like
Lemma~\ref{lemma:sum-of-square-roots-inverses}, were proved
in~\cite{Auer-Cesa-Bianchi-Gentile-2002} and~\cite{Jaksch-Ortner-Auer-2010}. The
main property of this inequality is that on the right-hand side $C$ does
\emph{not} multiply the $\sqrt{\sum_{t=1}^T a_t^2}$ term.

\begin{lemma}[Useful Inequality]
\label{lemma:useful}
Let $C, a_1, a_2, \dots, a_T \ge 0$. Then,
$$
\sum_{t=1}^T \min \left\{ \frac{a_t^2}{\sqrt{\sum_{i=1}^{t-1} a_i^2}}, \ C a_t \right\}
\le 3.5 \, C \max_{t=1,2,\dots,T} a_t \ + \ 3.5 \sqrt{\sum_{t=1}^T a_t^2} \; .
$$
\end{lemma}
\begin{proof}
Without loss of generality, we can assume that $a_t > 0$ for all $t$. Since otherwise we
can remove all $a_t = 0$ without affecting either side of the inequality. Let $M_t = \max\{a_1, a_2, \dots, a_t\}$ and $M_0 = 0$.
We prove that for any $\alpha > 1$
$$
\min\left\{ \frac{a_t^2}{\sqrt{\sum_{i=1}^{t-1} a_i^2}}, C a_t \right\}
\le 2 \sqrt{1+\alpha^2} \left( \sqrt{\sum_{i=1}^t a_i^2} - \sqrt{\sum_{i=1}^{t-1} a_i^2} \right) + \frac{C\alpha( M_t  - M_{t-1})}{\alpha - 1}
$$
from which the inequality follows by summing over $t=1,2,\dots,T$ and choosing $\alpha = \sqrt{2}$.
The inequality follows by case analysis. If $a_t^2 \le \alpha^2 \sum_{i=1}^{t-1} a_i^2$, we have
\begin{multline*}
\min\left\{ \frac{a_t^2}{\sqrt{\sum_{i=1}^{t-1} a_i^2}}, C a_t \right\}
\le \frac{a_t^2}{\sqrt{\sum_{i=1}^{t-1} a_i^2}}
= \frac{a_t^2}{\sqrt{\frac{1}{1+\alpha^2} \left( \alpha^2 \sum_{i=1}^{t-1} a_i^2 + \sum_{i=1}^{t-1} a_i^2 \right)}} \\
\le \frac{a_t^2\sqrt{1+\alpha^2}}{\sqrt{ a_t^2 + \sum_{i=1}^{t-1} a_i^2 }}
= \frac{a_t^2\sqrt{1+\alpha^2}}{\sqrt{\sum_{i=1}^t a_i^2}}
\le 2\sqrt{1+\alpha^2} \left( \sqrt{\sum_{i=1}^t a_i^2} - \sqrt{\sum_{i=1}^{t-1} a_i^2} \right)
\end{multline*}
where we have used $x^2/\sqrt{x^2+y^2} \le 2(\sqrt{x^2+y^2} - \sqrt{y^2})$ in the last step.
On the other hand, if $a_t^2 > \alpha^2 \sum_{t=1}^{t-1} a_i^2$, we have
\begin{multline*}
\min\left\{ \frac{a_t^2}{\sqrt{\sum_{i=1}^{t-1} a_i^2}}, \ C a_t \right\}
\le C a_t
= C \frac{\alpha a_t  - a_t}{\alpha - 1}
\le \frac{C}{\alpha - 1} \left( \alpha a_t  - \alpha \sqrt{\sum_{i=1}^{t-1} a_i^2} \right) \\
= \frac{C\alpha}{\alpha - 1} \left( a_t  - \sqrt{\sum_{i=1}^{t-1} a_i^2} \right)
\le \frac{C\alpha}{\alpha - 1} \left( a_t  - M_{t-1} \right)
= \frac{C\alpha}{\alpha - 1} \left( M_t  - M_{t-1} \right)
\end{multline*}
where we have used that $a_t = M_t$ and $\sqrt{\sum_{i=1}^{t-1} a_i^2} \ge M_{t-1}$.
\end{proof}

\begin{lemma}[{{\cite[Lemma~3.5]{Auer-Cesa-Bianchi-Gentile-2002}}}]
\label{lemma:sum-of-square-roots-inverses}
Let $a_1, a_2, \dots, a_T$ be non-negative real numbers. If $a_1 > 0$ then,
$$
\sum_{t=1}^T \frac{a_t}{\sqrt{\sum_{i=1}^t a_i}} \le 2 \sqrt{\sum_{t=1}^T a_t} \; .
$$
\end{lemma}
%
For completeness, a proof of Lemma~\ref{lemma:sum-of-square-roots-inverses} is
in~\ref{section:solo-ftrl-proofs}.

\begin{proof}[Proof of Theorem~\ref{theorem:regret-solo-ftrl}]
Let $\eta_t = \frac{1}{\sqrt{\sum_{i=1}^{t-1} \|\ell_i\|_*^2}}$, hence $R_t(w)
= \frac{1}{\eta_t} R(w)$.  We assume without loss of generality that
$\|\ell_t\|_* > 0$ for all $t$, since otherwise we can remove all rounds $t$
where $\ell_t = 0$ without affecting the regret and the predictions of the
algorithm on the remaining rounds.  By Lemma~\ref{lemma:generic-regret-bound},
\begin{align*}
\Regret_T(u)
& \le \frac{1}{\eta_{T+1}} R(u) + \sum_{t=1}^T \left( \Breg_{R_t^*}(-L_t, -L_{t-1}) - R_t^*(-L_t) + R_{t+1}^*(-L_t) \right) \; .
\end{align*}
We upper bound the terms of the sum in two different ways.
First, by Proposition~\ref{proposition:conjugate-properties}, we have
$$
\Breg_{R_t^*}(-L_t, -L_{t-1}) - R_t^*(-L_t) + R_{t+1}^*(-L_t)
\le \Breg_{R_t^*}(-L_t, -L_{t-1})
\le \frac{\eta_t \|\ell_t\|_*^2}{2\lambda} \; .
$$
Second, we have
\begin{align*}
\Breg_{R_t^*}&(-L_t, -L_{t-1}) - R_t^*(-L_t) + R_{t+1}^*(-L_t) \\
& = \Breg_{R_{t+1}^*}(-L_t, -L_{t-1}) + R^*_{t+1}(-L_{t-1}) - R_t^*(-L_{t-1}) \\
& \qquad + \langle \nabla R_t^*(-L_{t-1})-\nabla R_{t+1}^*(-L_{t-1}), \ell_t \rangle  \\
& \le \frac{\eta_{t+1} \|\ell_t\|_*^2}{2\lambda} + \| \nabla R_t^*(-L_{t-1})-\nabla R_{t+1}^*(-L_{t-1})\| \cdot \|\ell_t\|_* \\
& = \frac{\eta_{t+1} \|\ell_t\|_*^2}{2\lambda} + \| \nabla R^*(- \eta_{t} L_{t-1})-\nabla R^*(- \eta_{t+1} L_{t-1})\| \cdot \|\ell_t\|_* \\
& \le \frac{\eta_{t+1} \|\ell_t\|_*^2}{2\lambda} + \min\left\{\frac{1}{\lambda} \|L_{t-1}\|_* \left(\eta_{t} - \eta_{t+1} \right), D\right\} \|\ell_t\|_* \; ,
\end{align*}
where in the first inequality we have used the fact that $R^*_{t+1}(-L_{t-1})
\le R_t^*(-L_{t-1})$, H\"older's inequality, and
Proposition~\ref{proposition:conjugate-properties}.  In the second inequality
we have used properties 5 and 7 of
Proposition~\ref{proposition:conjugate-properties}. Using the definition of
$\eta_{t+1}$ we have
\begin{align*}
\frac{\|L_{t-1}\|_* (\eta_{t} -\eta_{t+1})}{\lambda}
\le \frac{ \|L_{t-1}\|_*}{\lambda \sqrt{\sum_{i=1}^{t-1} \|\ell_i\|_*^2}}
\le \frac{ \sum_{i=1}^{t-1} \|\ell_i\|_*}{\lambda \sqrt{\sum_{i=1}^{t-1} \|\ell_i\|_*^2}}
\le \frac{\sqrt{t-1}}{\lambda}
\le \frac{\sqrt{T-1}}{\lambda}.
\end{align*}
Denoting by $H=\min\left\{\frac{\sqrt{T-1}}{\lambda},D\right\}$ we have
\begin{align*}
& \Regret_T(u)
\le \frac{1}{\eta_{T+1}} R(u) + \sum_{t=1}^T \min\left\{ \frac{\eta_t \|\ell_t\|_*^2}{2\lambda}, \ H \|\ell_t\|_* + \frac{\eta_{t+1} \|\ell_t\|_*^2}{2\lambda}  \right\} \\
& \le \frac{1}{\eta_{T+1}} R(u) + \frac{1}{2\lambda} \sum_{t=1}^T  \eta_{t+1} \|\ell_t\|_*^2 + \frac{1}{2\lambda} \sum_{t=1}^T \min\left\{ \eta_t \|\ell_t\|_*^2, \ 2 \lambda H \|\ell_t\|_* \right\} \\
& = \frac{1}{\eta_{T+1}} R(u) + \frac{1}{2\lambda} \sum_{t=1}^T  \frac{\|\ell_t\|_*^2}{\sqrt{\sum_{i=1}^t \|\ell_i\|_*^2}} + \frac{1}{2 \lambda} \sum_{t=1}^T \min\left\{ \frac{\|\ell_t\|_*^2}{\sqrt{\sum_{i=1}^{t-1} \|\ell_i\|_*^2}}, \ 2 \lambda H \|\ell_t\|_* \right\} \; .
\end{align*}
We bound each of the three terms separately. By definition of $\eta_{T+1}$, the
first term is $\frac{1}{\eta_{T+1}} R(u) = R(u) \sqrt{\sum_{t=1}^T
\|\ell_t\|_*^2}$.  We upper bound the second term using
Lemma~\ref{lemma:sum-of-square-roots-inverses} as
$$
\frac{1}{2\lambda} \sum_{t=1}^T  \frac{\|\ell_t\|_*^2}{\sqrt{\sum_{i=1}^t \|\ell_i\|_*^2}}
\le \frac{1}{\lambda} \sqrt{\sum_{t=1}^T \|\ell_t\|_*^2} \; .
$$
Finally, by Lemma~\ref{lemma:useful} we upper bound the third term as
$$
\frac{1}{2 \lambda} \sum_{t=1}^T \min\left\{ \frac{\|\ell_t\|_*^2}{\sqrt{\sum_{i=1}^{t-1} \|\ell_i\|_*^2}}, \ 2 \lambda \|\ell_t\|_* H \right\}
\le 3.5 H \max_{t \le T} \|\ell_t\|_* + \frac{1.75}{\lambda} \sqrt{\sum_{t=1}^T \|\ell_t\|_*^2} \; .
$$
Putting everything together gives the stated bound.
\end{proof}
