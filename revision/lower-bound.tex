\section{Lower Bound}
\label{section:lower-bound}

We show a lower bound on the worst-case regret of any algorithm for OLO. The
proof, presented in~\ref{section:lower-bound-proof}, is a standard
probabilistic argument.

\begin{theorem}[Lower Bound]
\label{theorem:simple-lower-bound}
Let $K \subseteq V$ be any non-empty bounded closed convex subset. Let $D =
\sup_{u,v \in K} \|u - v\|$ be the diameter of $K$. Let $A$ be any (possibly
randomized) algorithm for OLO on $K$. Let $T$ be any non-negative integer and
let $a_1, a_2, \dots, a_T$ be any non-negative real numbers.  There exists a
sequence of vectors $\ell_1, \ell_2, \dots, \ell_T$ in the dual vector space
$V^*$ such that $\|\ell_1\|_* = a_1, \|\ell_2\|_* = a_2, \dots, \|\ell_T\|_* =
a_T$ and the regret of algorithm $A$ satisfies
\begin{equation}
\label{equation:simple-lower-bound}
\Regret_T \ge \frac{D}{\sqrt{8}} \sqrt{\sum_{t=1}^T\|\ell_t\|_*^2} \; .
\end{equation}
\end{theorem}

The upper bounds on the regret, which we have proved for our algorithms, have
the same dependency on the norms of the loss vectors.  However, a gap remains
between the lower bound and the upper bounds.

The upper bound on regret of \textsc{SOLO FTRL} is of
the form $O(\sqrt{\sup_{v \in K} f(v) \sum_{t=1}^T \|\ell_t\|_*^2})$ where $f$
is any $1$-strongly convex function with respect to $\|\cdot\|$.  The same
upper bound is also achieved by \textsc{FTRL} with a constant learning rate
when the number of rounds $T$ and $\sum_{t=1}^T \|\ell_t\|_*^2$ are known
upfront \cite[Chapter 2]{Shalev-Shwartz-2011}.  The lower bound is
$\Omega(D\sqrt{\sum_{t=1}^T \|\ell_t\|_*^2})$.

The gap between $D$ and $\sqrt{\sup_{v \in K} f(v)}$ can be substantial.  For
example, if $K$ is the probability simplex in $\R^d$ and $f(w) = \ln(d) +
\sum_{j=1}^d w_j \ln w_j$ is the shifted negative entropy, the
$\|\cdot\|_1$-diameter of $K$ is $2$, $f$ is non-negative and $1$-strongly
convex with respect to $\|\cdot\|_1$, but $\sup_{v \in K} f(v) = \ln(d)$.  On
the other hand, if the norm $\|\cdot\|_2 = \sqrt{\langle \cdot, \cdot \rangle}$
arises from an inner product $\langle \cdot, \cdot \rangle$, the lower bound
matches the upper bounds within a constant factor.  The reason is that for any
$K$ with $\|\cdot\|_2$-diameter $D$, the function $f(w) = \frac{1}{2} \|w -
w_0\|_2^2$, where $w_0$ is an arbitrary point in $K$, is $1$-strongly convex
with respect to $\|\cdot\|_2$ and satisfies that $\sqrt{\sup_{v \in K} f(v)}
\le D$. This leads to the following open problem (posed also
in~\cite{Kwon-Mertikopoulos-2014}):
%
\begin{quotation}
\noindent
\emph{Given a bounded convex set $K$ and a norm $\|\cdot\|$, construct a
non-negative function $f:K \to \R$ that is $1$-strongly convex with respect to
$\|\cdot\|$ and minimizes $\sup_{v \in K} f(v)$.}
\end{quotation}
%
As shown in~\cite{Srebro-Sridharan-Tewari-2011}, the existence of $f$ with
small $\sup_{v \in K} f(v)$ is equivalent to the existence of an algorithm for
OLO with $\widetilde O(\sqrt{T \sup_{v \in K} f(v)})$ regret assuming
$\|\ell_t\|_* \le 1$.  The $\widetilde O$ notation hides a polylogarithmic
factor in $T$.
