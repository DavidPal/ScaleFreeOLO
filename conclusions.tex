\section{Conclusions}
\label{section:conclusions}

We have investigated scale-free algorithms for online linear optimization and
we have shown that scale-free property leads to algorithms which have optimal
regret and do not need to know or assume \textbf{anything} about the sequence
of loss vectors. In particular, the algorithms do not assume any upper or lower
bounds on the norms of the loss vectors or the number of rounds.

We have designed two scale-free algorithms based on \textsc{Follow The
Regularizer Leader}.  Their regret is $O \left(\sqrt{\sup_{v \in K} f(v)
\sum_{t=1}^T \|\ell\|_*^2} \right)$ where $f$ is any non-negative $1$-strongly
convex function with respect to a norm $\|\cdot\|$ defined on the decision set
and where $\|\cdot\|_*$ is the dual norm to $\|\cdot\|$.

Similar, but weaker result holds for a scale-free algorithm based on
\textsc{Mirror Descent}. However, we have also shown this algorithm is strictly
weaker than algorithms based on \textsc{Follow The Regularizer Leader}. Namely,
we gave examples of regularizers for which the scale-free version of
\textsc{Mirror Descent} has $\Omega(T)$ regret or worse.

We have shown an $\frac{D}{\sqrt{8}} \sqrt{\sum_{t=1}^T \|\ell\|_*^2}$
lower bound on regret of any algorithm for decision set with diameter $D$.

We have shown that one of the scale-free \textsc{FTRL}-based the algorithms has
regret $O \left(f(u) \sqrt{\sum_{t=1}^T \|\ell_t\|_*^2} + \max_{t=1,2,\dots,T}
\|\ell_t\|_* \sqrt{T} \right)$ with respect to any competitor $u \in K$, where
$f$ is any non-negative $1$-strongly convex function defined on the decision
set.  The latter result makes sense even when the decision set $K$ is
unbounded. Notice that with regularizer $f(w) = \frac{1}{2}\|w\|_2^2$ the
regret depends on norm of the competitor $\|u\|_2$ is quadratically. There
exist non-scale-free algorithms \cite{ McMahan-Streeter-2012,
McMahan-Abernethy-2013, Orabona-2013, McMahan-Orabona-2014, Orabona-2014} that
have only $O(\|u\|_2 \sqrt{\log \|u_2\|_2})$ or $O(\|u\|_2 \log \|u\|_2)$
dependency.  These algorithms, however, assume an a priori bound on the norm of
the loss vectors.  This leads to another open problem:
%
\begin{quotation}
\noindent
\emph{Design a scale-free algorithm that has $O(\|u\|_2 \sqrt{\log \|u\|_2})$
dependency on the norm $\|u\|_2$ of the competitor.}
\end{quotation}
